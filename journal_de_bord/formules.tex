%! Author = jules
%! Date = 17/02/24

% Preamble
\documentclass[11pt]{article}

% Packages
\usepackage{amsmath}

% Document
\begin{document}
    \title{TIPE}
    \author{Jules et Noa}
    \maketitle
    \section{Tension aux bornes des dipôles}\label{sec:i}

    \begin{itemize}
    \item La charge du condensateur (circuit RC) en fonction de \(E\) (tension du générateur) en Volts, \(U_{0}\) la tension initiale du condensateur en Volts, \(R\) la résistance en Ohm, et \(C\) la capacité en Farads est :
    \[
        U_{c}(t) = (U_{0}-E) \times e^{\frac{-t}{\tau}} + E
    \]
         où \(\tau = RC\) désigne la constante de temps en secondes.
        \item TODO : créer un algo pour générer les formules pour un circuit quelconque.

\end{itemize}
    \section{Avantages (valeur ajoutéee) du logiciel}
    \begin{itemize}
        \item Open Source et gratuit
        \item Multiplatforme (windows et linux)
        \item Correction du circuit
        \item Mesure de la tension aux bornes des dipôles et affichage graphique
        \item
\end{itemize}
    \section{TODO}
    \begin{itemize}
        \item Création d'algo pour les équa diff (Jules)
        \item Algo/structure de donnée (lsc) (Noa)
        \item Correction (Noa) (court-circuit, masse)
        \item GUI (Jules (pas priorité))
        \item Création des graphs (Jules)
        \item Tests concurrence
        \item Journal (latex) (git)
    \end{itemize}
    \section{21/03/2024}
    Définition (court-circuit) : Un circuit est court-circuité s'il existe
    deux branches en dérivation dont l'une est de résistance très faible devant celle de l'autre (Masse ?)
    \\
    11h00 http://abcsite.free.fr/physique/elec/elch5.html
    \\
    10h30 https://ieeexplore-ieee-org.ezproxy.uphf.fr/xpl/tocresult.jsp?isnumber=23549

    \section{04/04/2024}
    https://fr.wikipedia.org/wiki/Topologie \\
   Notes sur la théorie des graphes et les circuits (à formaliser): \\
   Dipôles = étiquettes des arrêtes \\
   Mailles = arrêtes \\
   Noeuds = sommets \\
   Direction du courant = orientation du graphe \\
   
   Arbre = graphe sans cycle fermée. \\
   Donc pas de courant dans un arbre. \\
   Graphe de réseau = ensemble d'arbres. \\
   Lien = mailles retirées pour former l'arbre. \\
   Brindilles = mailles restantes dans l'arbre. \\
   Pour un graphe à n nœuds, le nombre de mailles dans chaque arbre, t, doit être : 
   $\displaystyle t=n-1 $ et \\ 
   $\displaystyle b=\ell +t $ \\
   où b est le nombre de mailles du graphe et l, le nombre de liens 
   retires pour former l'arbre. \\
   Résolution d'un circuit : \\
   But : trouver des variables indépendantes qui caractérisent tout les dipôles d'un circuit (Par exemple : Intensité). \\
   Le nombre minimum de courants de maille requis pour une solution complète est l'ensemble des liens noté I. \\
   Pour résoudre : ensemble de boucles indépendantes. \\ (Voir https://en.wikipedia.org/wiki/Mesh-analysis) (remplacer - par underscore) \\


   



    \end{document}
